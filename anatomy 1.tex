\documentclass[4pt]{article}

\usepackage{graphicx}
\graphicspath{{image/}}

\title{my}

\begin{document}


\centering\huge\texttt{NATIONAL INSTITUTE OF TECHNOLOGY, RAIPUR}\\

\begin{figure}[h]
\centering
\includegraphics[scale=1.5]{NITRR.jpg}
\end{figure}
\begin{center}


\centering\huge\texttt{ASSIGNMENT\\ ON\\ INTRODUCTION TO HUMAN BODY}\\
\end{center}



\begin{minipage}[t]{5cm}
\flushleft\Large\textbf{\underline{submitted by:}}\\
Itisha Kaiwart\\
Roll No.-21111024
\end{minipage}
\hfill
\begin{minipage}[t]{5cm}
\Large\textbf{\underline{under the supervision of:}}\\
Dr. Saurabh Gupta
\end{minipage}

\newpage

\Large\flushleft\textbf{Anatomy}(ana-=up;tomy-=process of cutting)is the science of body structure and relationship among them.\\
It was first studied by \textbf{dissection}(dis-=apart;-section=act of cutting).\\
\textbf{Physiology}(physio-=nature;-logy=study of)is the science of body function-how the body parts work.

\section{\huge\textbf{LEVELS OF STRUCTURAL ORGANISATION}}
There are in total 6 level of structural organisation in our body.\\
1. Chemical level- This is very basic level of organisation that include the smallest unit called atom.Certain atoms, 
such as carbon (C), hydrogen (H), oxygen (O), nitrogen (N), 
phosphorus (P), calcium (Ca), and sulfur (S), are essential for maintaining life.
example; through DNA gentic information passes from one generation to next.

2. cellular level-there are many kind of cell in our body-muscle cell,nerve cell,epithelial cell.\\

3. Tissue level- Cells combine to form tissue.\\

4. Organ level- these are structure that are made up of two or more kind of tissue.they have specific function and have recognisable shape.\\
5.  System (organ-system)- level. A system (or chapter, in our 
language analogy) consists of related organs (paragraphs) with a 
common function.


\section{\huge\textbf{HOMOEOSTASIS}}
homoeostasis is the maintenance of relatively stable condition in our body.\\
The fluid inside the cell is intracellular fluid.The fluid outside the cell is called extracellular fluid.The ECF that fills the narrow 
spaces between cells of tissues is known as interstitial fluid.: ECF 
within blood vessels is termed blood plasma, within lymphatic vessels 
it is called lymph, in and around the brain and spinal cord it is known as 
cerebrospinal fluid, in joints it is referred to as synovial fluid, and the 
ECF of the eyes is called aqueous humor and vitreous body.\\
\textbf{control of homoeostasis}\\
Disruptions of homeostasis come from external and internal stimuli and 
 psychological stresses. When disruption of homeostasis is mild and temporary, 
responses of body cells quickly restore balance in the internal environment. If 
disruption is extreme, regulation of homeostasis may fail.
, the nervous and endocrine systems acting together or separately regulate homoeostasis. The nervous system detects body changes and 
sends nerve impulses to counteract changes in controlled conditions. The 
 endocrine system regulates homeostasis by secreting hormones.
 
 \textbf{Feedback system}\\
 There are 3 component in a feed back system .
 1. Receptor: Which detect the change in control condition and sends input to control centre.(afferent pathway)
 2. Control centre:evaluate the input and generate output.
 3. Effector: recieves output from control centre.(efferent pathway)
 
 There are 2 types of feedback system;\\
 1. positive feedback system; a 
positive feedback system tends to strengthen or reinforce a change .
in one of the body's controlled conditions.\\
example- during child birth
2. Negative Feedback system;A negative feedback system reverses
a change in a controlled condition.example-regulation of blood pressure.
\section{\huge\textbf{BASIC ANATOMICAL TERMINOLOGY}}
\textbf{Body position}
If the body is lying facedown, it is in the prone position. If the 
body is lying faceup, it is in the supine position.\\
\textbf{Regional names}\\
The human body is divided into several major regions that can be 
identified externally. The principal regions are the head, neck, trunk, 
upper limbs, and lower limbs.\\
\textbf{Planes and section}\\
 A sagittal 
plane  is a vertical plane that divides the body or an organ into right and left sides.\\
A midsagittal planes passes through the midline and divide the oragan into equal halfs.and opposite of that is known as parasagittal plane.\\
\textbf{Body cavities}\\
Body cavities are spaces that enclose internal organs. Bones, muscles, 
ligaments, and other structures separate the various body cavities from 
one another



 
\section{\huge\textbf{MEDICAL IMAGING }}
Medical imaging refers to techniques and procedures used to create images 
of the human body. They allow visualization of internal structures to diagnose 
abnormal anatomy and deviations from normal physiology.
Common imaging techniques are;\\
Radiography\\
Magnetic Resonance Imaging\\
Computed Tomography\\
CORONARY (CARDIAC) COMPUTED TOMOGRAPHY 
ANGIOGRAPHY (CCTA) SCAN\\
POSITRON EMISSION TOMOGRAPHY (PET)\\
ENDOSCOPY\\



















\end{document}